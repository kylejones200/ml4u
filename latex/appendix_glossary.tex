\section{Glossary}\label{glossary}

\subsection{A}\label{a}

\textbf{Anomaly Detection}\\
A machine learning technique that identifies unusual patterns or outliers in data that deviate from normal behavior. In utilities, used to detect equipment malfunctions, cybersecurity threats, or operational irregularities.

\textbf{ARIMA (AutoRegressive Integrated Moving Average)}\\
A time series forecasting model that captures autoregressive patterns, trends (through differencing), and moving average components. Commonly used for load forecasting.

\textbf{Asset Management}\\
The systematic process of managing physical infrastructure assets (transformers, breakers, lines) throughout their lifecycle, from installation to replacement.

\subsection{C}\label{c}

\textbf{Classification}\\
A machine learning task that predicts discrete categories or labels (e.g., ``healthy'' vs.~``failure-prone'' for equipment).

\textbf{Clustering}\\
An unsupervised machine learning technique that groups similar observations together without predefined labels. Used for customer segmentation and asset grouping.

\textbf{Computer Vision}\\
A field of artificial intelligence that enables machines to interpret and understand visual information from images or video. Used in utilities for automated infrastructure inspections.

\subsection{D}\label{d}

\textbf{Data Drift}\\
Changes in the distribution of input data over time that can degrade model performance. Requires monitoring and periodic retraining.

\textbf{Data Lake}\\
A centralized repository that stores raw data in its native format. Enables analytics on diverse data types without predefined schemas.

\textbf{Demand Response}\\
Programs that encourage customers to reduce or shift electricity consumption during peak periods, helping utilities manage demand without building new generation capacity.

\textbf{DER (Distributed Energy Resources)}\\
Small-scale power generation or storage systems located near where energy is consumed, such as rooftop solar panels, battery storage, or small wind turbines.

\textbf{Distribution Grid}\\
The portion of the electric grid that delivers power from substations to end customers (homes, businesses).

\subsection{E}\label{e}

\textbf{EAM (Enterprise Asset Management)}\\
Software systems that track and manage physical assets throughout their lifecycle, including maintenance records, inspections, and work orders.

\textbf{Ensemble Methods}\\
Machine learning techniques that combine predictions from multiple models to improve accuracy and robustness.

\textbf{ETL (Extract, Transform, Load)}\\
The process of extracting data from source systems, transforming it into a usable format, and loading it into a target system for analysis.

\subsection{F}\label{f}

\textbf{Feature Engineering}\\
The process of creating new input variables (features) from raw data to improve model performance. Examples include creating time-based features (hour of day, day of week) or interaction terms.

\textbf{Forecast Horizon}\\
The time period into the future that a forecast predicts (e.g., 1 hour ahead, 24 hours ahead, 1 week ahead).

\subsection{G}\label{g}

\textbf{GIS (Geographic Information System)}\\
Systems that capture, store, and analyze geographic and spatial data. Used in utilities to map infrastructure locations and analyze geospatial relationships.

\textbf{Grid}\\
The interconnected network of power generation, transmission, and distribution systems that deliver electricity from producers to consumers.

\subsection{L}\label{l}

\textbf{Load Forecasting}\\
The process of predicting future electricity demand. Critical for generation scheduling, market participation, and grid operations.

\textbf{LSTM (Long Short-Term Memory)}\\
A type of recurrent neural network designed to capture long-term dependencies in time series data. Used for load and renewable generation forecasting.

\subsection{M}\label{m}

\textbf{MAPE (Mean Absolute Percentage Error)}\\
A forecast accuracy metric that expresses error as a percentage of actual values. Commonly used in load forecasting.

\textbf{MLOps (Machine Learning Operations)}\\
Practices and tools for deploying, monitoring, and maintaining machine learning models in production environments.

\textbf{Model Drift}\\
Degradation in model performance over time due to changes in data distribution or underlying relationships. Requires monitoring and retraining.

\subsection{N}\label{n}

\textbf{Net Load}\\
The difference between total electricity demand and behind-the-meter generation (e.g., rooftop solar). What the grid must actually supply.

\textbf{NLP (Natural Language Processing)}\\
A branch of artificial intelligence that enables computers to understand, interpret, and generate human language. Used in utilities to analyze maintenance logs and compliance documents.

\subsection{O}\label{o}

\textbf{Orchestration}\\
The automated coordination of multiple tasks or workflows, managing dependencies, scheduling, and error handling.

\textbf{Outage}\\
An interruption in electric service to customers, caused by equipment failures, weather, or other factors.

\textbf{Overfitting}\\
When a machine learning model learns patterns specific to training data that don't generalize to new data, resulting in poor performance on unseen examples.

\subsection{P}\label{p}

\textbf{Predictive Maintenance}\\
A maintenance strategy that uses data and analytics to predict when equipment failures are likely to occur, enabling proactive intervention before failures happen.

\textbf{Precision}\\
A classification metric measuring the proportion of positive predictions that are actually correct. Important when false alarms are costly.

\textbf{PMU (Phasor Measurement Unit)}\\
High-speed sensors that measure voltage and current phasors across the grid, providing real-time visibility into grid conditions.

\subsection{R}\label{r}

\textbf{Recall}\\
A classification metric measuring the proportion of actual positives that are correctly identified. Important when missing failures is costly.

\textbf{Regression}\\
A machine learning task that predicts continuous numerical values (e.g., load in megawatts, temperature in degrees).

\textbf{Reinforcement Learning}\\
A machine learning paradigm where an agent learns to make decisions by interacting with an environment and receiving rewards or penalties.

\textbf{Reliability Metrics}\\
Measures of grid performance, such as SAIDI (System Average Interruption Duration Index) and SAIFI (System Average Interruption Frequency Index).

\subsection{S}\label{s}

\textbf{SCADA (Supervisory Control and Data Acquisition)}\\
Systems that monitor and control grid equipment in real time, collecting telemetry data and enabling remote control of switches, breakers, and other devices.

\textbf{SARIMA (Seasonal ARIMA)}\\
An extension of ARIMA that explicitly models seasonal patterns, essential for load forecasting which has strong daily and weekly cycles.

\textbf{SHAP (SHapley Additive exPlanations)}\\
A method for explaining individual model predictions by quantifying each feature's contribution to the prediction.

\textbf{Supervised Learning}\\
Machine learning where models learn from labeled training data (inputs paired with known outputs).

\subsection{T}\label{t}

\textbf{Time Series}\\
Data collected over time at regular intervals (e.g., hourly load, daily temperature). Requires specialized modeling techniques to capture temporal patterns.

\textbf{Transformer}\\
Electrical equipment that changes voltage levels in the power grid. Also refers to a type of neural network architecture used in natural language processing.

\textbf{Transmission Grid}\\
The high-voltage portion of the electric grid that moves power over long distances from generation plants to distribution substations.

\subsection{U}\label{u}

\textbf{Unsupervised Learning}\\
Machine learning where models find patterns in data without labeled examples. Includes clustering and anomaly detection.

\subsection{V}\label{v}

\textbf{Voltage Control}\\
The process of maintaining voltage levels within acceptable limits across the grid, typically through reactive power management.

\subsection{W}\label{w}

\textbf{Work Order}\\
A formal request for maintenance, repair, or inspection of equipment, typically tracked in EAM systems.

\documentclass[11pt,letterpaper]{article}

\usepackage[utf8]{inputenc}
\usepackage[T1]{fontenc}
\usepackage[margin=1in]{geometry}
\usepackage{enumitem}
\usepackage{url}
\usepackage{hyperref}

\title{Manning Book Proposal\\
Machine Learning for Power \& Utilities}
\author{Drew Triplett, David Radford, and Kyle T. Jones}
\date{\today}

\begin{document}

\maketitle

\section*{Book Title}
Machine Learning for Power \& Utilities

\section*{Name of Author(s)}
Drew Triplett, David Radford, and Kyle T. Jones

\section{Tell us about yourself}

\subsection*{What are your qualifications for writing this book?}
I have extensive hands-on experience applying machine learning to electric utility operations, including load forecasting, predictive maintenance, outage prediction, and grid optimization. I've worked directly with utilities to deploy ML models into production, integrating them with SCADA, GIS, and asset management systems. My background combines deep technical expertise in machine learning with practical understanding of utility operations, regulatory requirements, and the unique challenges of critical infrastructure.

\subsection*{Do you have any unique characteristics or experiences that will make you stand out as the author?}
I've seen firsthand the gap between ML research and utility operations—the ``pilot purgatory'' where models never make it to production. This book is written from the perspective of someone who has deployed ML in control rooms, worked with operators and engineers, and understands both the technical requirements and the business constraints. I've worked with real utility data (eGrid, SCADA, smart meters) and understand the data quality, integration, and regulatory challenges that other ML books ignore. The book includes production-ready code examples that demonstrate enterprise integration patterns, not just toy datasets.

\section{Tell us about the book's topic}

\subsection*{What is the technology or idea that you're writing about?}
This book teaches machine learning and AI applications specifically for electric power utilities. It covers forecasting (load, renewable generation), predictive maintenance, outage prediction, computer vision for asset inspection, NLP for maintenance logs, LLMs for operational support, cybersecurity analytics, and MLOps for production deployment. The focus is on practical, production-ready implementations that integrate with existing utility systems (SCADA, GIS, EAM).

\subsection*{Why is it important now?}
Electric utilities face unprecedented challenges: aging infrastructure, electrification-driven demand growth, distributed energy resources creating grid variability, extreme weather increasing outages, and regulatory pressure for better reliability and transparency. Traditional deterministic models and manual processes cannot keep up. Meanwhile, utilities generate massive amounts of data (SCADA, smart meters, PMUs, asset records) that sits unused in silos. Machine learning can unlock this data to enable predictive operations, but there's a critical gap: most ML resources are generic and don't address utility-specific challenges like regulatory compliance, integration with legacy systems, or the need for explainability in critical infrastructure.

\subsection*{In a couple sentences, tell us roughly how it works or what makes it different from its alternatives?}
This book bridges the gap between ML theory and utility operations. Unlike generic ML books, it addresses utility-specific challenges: integrating with SCADA and GIS systems, handling regulatory requirements for model auditability, working with real utility data formats, and deploying models that operators and engineers can trust. Each chapter starts with a business problem, shows how ML solves it, and provides production-ready Python code that demonstrates enterprise integration patterns. The book progresses from foundational analytics (forecasting, classification) to advanced topics (LLMs, computer vision) while maintaining focus on operational deployment.

\section{Tell us about the book you plan to write}

\subsection*{What will the reader be able to do after reading this book?}

After reading this book, the reader will be able to:

\begin{itemize}
\item Build production-ready ML models for utility use cases: load forecasting, predictive maintenance, outage prediction, renewable generation forecasting, and customer analytics
\item Integrate ML models with existing utility systems (SCADA, GIS, EAM, customer systems) using APIs, databases, and message queues
\item Apply computer vision to automate asset inspections using drone imagery and pole-mounted cameras
\item Use NLP to extract insights from maintenance logs, work orders, and regulatory documents
\item Deploy LLMs for operational support, including summarization, entity extraction, and multimodal AI for field operations
\item Implement MLOps workflows using MLflow for model tracking, automated retraining, and production deployment
\item Build feature engineering pipelines for utility data, including temporal features, geospatial features, and grid topology features
\item Apply causal inference methods to evaluate policy impacts and program effectiveness
\item Use multi-task learning to simultaneously predict multiple related outcomes (e.g., CO2, NOx, SO2 emissions)
\item Measure ROI of ML initiatives using frameworks like CPMAI
\item Ensure models meet regulatory requirements for explainability, fairness, and auditability
\item Deploy real-time analytics pipelines for control room integration
\item Build enterprise-scale ML platforms that support multiple use cases across the utility
\end{itemize}

\subsection*{Is your book designed to teach a topic or to be used as a reference?}
The book is designed primarily to teach, with a progressive structure that builds from fundamentals to advanced topics. Each chapter includes hands-on Python examples that readers can run and modify. However, the code examples and integration patterns are also designed to serve as reference material for practitioners implementing similar solutions.

\subsection*{Does this book fall into a Manning series such as In Action, In Practice, Month of Lunches, or Grokking?}
This book fits best in the ``In Action'' series, as it provides comprehensive, practical coverage of ML applications in utilities with real-world examples and production-ready code.

\subsection*{Are there any unique characteristics of the proposed book, such as a distinctive visual style, questions and exercises, supplementary online materials like video, etc?}
The book includes:
\begin{itemize}
\item Production-ready Python code examples for each use case (33+ code files)
\item Real utility data examples (eGrid power plant data, SCADA simulations, smart meter data)
\item Integration patterns showing how ML connects to enterprise systems
\item Business context for each technique, explaining why it matters operationally
\item Regulatory and compliance considerations throughout
\item Case studies from real utility deployments
\item Minimalist, clean visualizations (no chart junk)
\item Code that follows PEP 8 and production best practices
\end{itemize}

\section{Q\&A}

\subsection*{What are the three or four most commonly-asked questions about this technology?}

\begin{itemize}
\item How do I get started with ML in utilities if I don't have a data science background?
\item How do I integrate ML models with existing SCADA, GIS, and asset management systems?
\item What are the regulatory requirements for deploying ML in critical infrastructure, and how do I ensure models are explainable and auditable?
\item How do I measure ROI and justify ML investments to utility leadership?
\item What's the difference between a pilot project and production deployment, and how do I bridge that gap?
\item How do I handle the data quality and integration challenges when working with utility data from multiple siloed systems?
\end{itemize}

\section{Tell us about your readers}

Your book will teach your readers how to accomplish the objectives you've established for the book. It's critical to be clear about the minimum qualifications you're assuming of your reader and what you'll need to teach them.

\subsection*{What skills do you expect the minimally-qualified reader to already have? Be specific.}

\begin{itemize}
\item Intermediate Python programming skills (comfortable with pandas, numpy, basic object-oriented programming)
\item Basic understanding of machine learning concepts (supervised vs unsupervised learning, training vs testing)
\item Familiarity with utility operations (understanding of load forecasting, asset management, or grid operations)
\item Basic knowledge of data analysis (working with CSV files, understanding data types, basic statistics)
\item No deep expertise in specific ML algorithms required (we teach the algorithms as needed)
\item No prior experience with utility-specific ML applications required
\end{itemize}

\subsection*{What are the typical job roles for the primary reader? Be specific:}

\begin{itemize}
\item Data scientists or analysts working in utilities or energy companies (2-5 years experience)
\item Utility engineers or operations staff looking to add ML capabilities to their skill set
\item Software engineers building analytics platforms for utilities
\item Consultants working with utility clients on ML projects
\item Graduate students or researchers focusing on energy systems and ML
\end{itemize}

\subsection*{What will motivate the reader to learn this topic?}
Readers are motivated by the urgent need to modernize utility operations in the face of aging infrastructure, electrification, and renewable integration. They see the potential of ML but struggle with the gap between generic ML tutorials and utility-specific requirements. They need practical, production-ready examples that show how to integrate ML with existing systems and meet regulatory requirements. They want to move beyond pilot projects to operational deployment that delivers measurable business value.

\section{Tell us about the competition and the ecosystem}

\subsection*{What are the best books available on this topic and how does the proposed book compare to them?}
Most ML books are generic and don't address utility-specific challenges. ``Hands-On Machine Learning'' by Aurélien Géron is excellent for ML fundamentals but doesn't cover utility use cases or integration patterns. ``Applied Machine Learning'' by David Forsyth covers applications but not utilities. There are academic texts on power systems and ML, but they lack practical implementation details. This book is unique in providing production-ready code for utility-specific ML applications with enterprise integration patterns.

\subsection*{What are the best videos available on this topic and how does the proposed book compare to them?}
Online courses (Coursera, edX) cover ML fundamentals but not utility applications. YouTube has scattered tutorials on load forecasting or predictive maintenance, but nothing comprehensive. This book provides a structured, comprehensive curriculum with hands-on code examples that videos cannot match.

\subsection*{What other resources would you recommend to someone wanting to learn this subject?}
IEEE Power \& Energy Society publications, EPRI reports, utility industry conferences (DistribuTECH, IEEE PES), and open-source projects like GridLAB-D. This book complements these resources by providing practical implementation guidance.

\subsection*{What are the most important web sites and companies?}
EPRI (Electric Power Research Institute), IEEE Power \& Energy Society, NREL (National Renewable Energy Laboratory), utility industry publications (Utility Dive, Energy Central), and ML platforms (Databricks, AWS, Azure) with utility-specific solutions.

\subsection*{Where do others interested in this topic gather?}
IEEE PES conferences, DistribuTECH, utility industry forums, LinkedIn groups for utility data science, and professional associations like the Association of Energy Engineers.

\section{Book size and illustrations}

Please estimate:

\subsection*{The approximate number of published pages to within a 50-page range}
Approximately 200-250 published pages (current manuscript is approximately 210 pages, with 28 chapters plus appendices)

\subsection*{The approximate number of diagrams and other graphics}
Approximately 40-60 diagrams and graphics, including:
\begin{itemize}
\item Architecture diagrams showing ML integration with utility systems
\item Flowcharts for ML workflows
\item Visualizations of model outputs (forecasts, predictions, anomaly detections)
\item Code execution results and data visualizations
\item System integration diagrams
\end{itemize}

\subsection*{The approximate number of code listings}
Approximately 100-150 code listings, including:
\begin{itemize}
\item 33+ complete Python scripts (one per chapter)
\item Code snippets demonstrating specific techniques
\item Configuration files (YAML, JSON)
\item API integration examples
\item Database queries and data processing pipelines
\end{itemize}



\section{Schedule}

\subsection*{To write and revise a chapter, most authors require 2-4 weeks. Please estimate your writing schedule}

\textbf{Chapter 1:} We typically expect the first chapter in about 1 month\\
{February 2026}

\textbf{1/3 manuscript:}\\
{[May 2026 - approximately 3-4 months after Chapter 1]}

\textbf{2/3 manuscript:}\\
{[September 2026 - approximately 6-8 months after Chapter 1]}

\textbf{3/3 manuscript:}\\
{[December 2026 - approximately 9-12 months after Chapter 1]}

\subsection*{Are there any critical deadlines for the completion of this book? New software versions? Known competition? Technical conferences?}
The utility industry is rapidly adopting ML, and there's growing competition from consulting firms and platform vendors. The book should be completed before major industry conferences (DistribuTECH typically in February/March) to maximize impact. There are no critical software version dependencies, as the book uses stable, widely-adopted libraries (scikit-learn, pandas, TensorFlow, etc.).

\section{Table of Contents}

The table of contents is your plan for teaching your intended readers the skills they need to accomplish the objectives you've established for the book. While every Table of Contents is different, there are a few common best practices for a typical In Action book.

\subsection*{Part 1: Foundations}

\begin{enumerate}
\item Introduction to Machine Learning in Power and Utilities
\begin{enumerate}
\item The Three Forces Disrupting Utilities
\item Why Machine Learning Matters
\item Data as a Strategic Asset
\item A Simple Example: Temperature-to-Load Forecasting
\item What This Book Covers
\item Summary
\end{enumerate}

\item Data Sources and Integration Patterns
\begin{enumerate}
\item Utility Data Landscape
\item SCADA and Real-Time Telemetry
\item Advanced Metering Infrastructure (AMI)
\item Asset Management Systems (EAM)
\item Geographic Information Systems (GIS)
\item External Data Sources (Weather, Markets)
\item Data Integration Patterns
\item Summary
\end{enumerate}

\item Machine Learning Fundamentals
\begin{enumerate}
\item Regression for Continuous Predictions
\item Classification for Categorical Outcomes
\item Clustering for Pattern Discovery
\item Model Evaluation Metrics
\item Utility Use Case Mapping
\item Summary
\end{enumerate}
\end{enumerate}

\subsection*{Part 2: Core Applications}

\begin{enumerate}
\setcounter{enumi}{3}
\item Load Forecasting and Demand Analytics
\begin{enumerate}
\item The Business Problem: Balancing Supply and Demand
\item Time Series Fundamentals: ARIMA Models
\item Weather-Driven Features
\item LSTM Neural Networks for Load Forecasting
\item Ensemble Methods
\item Production-Grade Forecasting: Feature Engineering and Multi-Tier Models
\item Forecast Accuracy Metrics and Operational Impact
\item Summary
\end{enumerate}

\item Predictive Maintenance for Grid Assets
\begin{enumerate}
\item The Cost of Unplanned Failures
\item Classification Models for Failure Prediction
\item Anomaly Detection for Condition Monitoring
\item Handling Class Imbalance
\item Risk Scoring and Prioritization
\item Integration with Asset Management Systems
\item Summary
\end{enumerate}

\item Outage Prediction and Storm Response
\begin{enumerate}
\item The Business Problem: Proactive Outage Management
\item Weather Data Integration
\item Feeder-Level Outage Risk Models
\item Vegetation and Asset Condition Features
\item Crew Staging Optimization
\item Real-Time Storm Analytics
\item Summary
\end{enumerate}

\item Grid Optimization and Operations
\begin{enumerate}
\item Voltage Optimization
\item Feeder Reconfiguration
\item Demand Response Optimization
\item Reinforcement Learning for Grid Control
\item Summary
\end{enumerate}

\item Renewable Integration and DER Forecasting
\begin{enumerate}
\item The Challenge of Variable Generation
\item Solar Generation Forecasting
\item Wind Power Forecasting
\item Net Load Forecasting with Behind-the-Meter Solar
\item Financial Modeling for Renewable Projects
\item Summary
\end{enumerate}

\item Customer Analytics and Engagement
\begin{enumerate}
\item Customer Segmentation
\item Load Profile Analysis
\item Demand Response Participation Prediction
\item Customer Churn and Satisfaction
\item Summary
\end{enumerate}
\end{enumerate}

\subsection*{Part 3: Advanced Techniques}

\begin{enumerate}
\setcounter{enumi}{9}
\item Computer Vision for Asset Inspection
\begin{enumerate}
\item Automating Visual Inspections
\item Pole and Line Detection
\item Vegetation Encroachment Detection
\item Solar Panel Defect Detection
\item Integration with Field Operations
\item Summary
\end{enumerate}

\item Natural Language Processing for Utilities
\begin{enumerate}
\item Text Classification for Maintenance Logs
\item Named Entity Recognition
\item Document Summarization
\item Regulatory Compliance Analysis
\item Summary
\end{enumerate}

\item Large Language Models and Multimodal AI
\begin{enumerate}
\item LLMs for Operational Support
\item Summarization and Entity Extraction
\item Multimodal AI for Field Operations
\item Prompt Engineering for Utility Domains
\item Hallucination and Safety Considerations
\item Summary
\end{enumerate}
\end{enumerate}

\subsection*{Part 4: Enterprise Integration and Operations}

\begin{enumerate}
\setcounter{enumi}{12}
\item Enterprise Integration Patterns
\begin{enumerate}
\item API-Based Integration
\item Database Replication and ETL
\item Message Queues and Streaming
\item GIS Integration
\item SCADA Integration
\item Summary
\end{enumerate}

\item MLOps for Utilities
\begin{enumerate}
\item The Pilot Purgatory Problem
\item Experiment Tracking with MLflow
\item Model Versioning and Registry
\item Automated Retraining Pipelines
\item Model Deployment as APIs
\item Model Monitoring and Drift Detection
\item Regulatory Compliance and Auditability
\item Summary
\end{enumerate}

\item Orchestration and Workflow Management
\begin{enumerate}
\item Task Orchestration with Prefect
\item Scheduling and Dependencies
\item Error Handling and Retries
\item State Management
\item Summary
\end{enumerate}

\item Platform Deployment and Architecture
\begin{enumerate}
\item Cloud vs On-Premise Considerations
\item Containerization and Kubernetes
\item Data Lake Architecture
\item Security and Access Control
\item Summary
\end{enumerate}
\end{enumerate}

\subsection*{Part 5: Specialized Topics}

\begin{enumerate}
\setcounter{enumi}{16}
\item Cybersecurity Analytics
\begin{enumerate}
\item Anomaly Detection for Network Security
\item Supervised Learning for Threat Detection
\item False Positive Management
\item Adversarial Machine Learning
\item Explainability for Security Teams
\item Summary
\end{enumerate}

\item AI Ethics and Regulatory Compliance
\begin{enumerate}
\item Fairness Audits
\item Explainability Requirements
\item Regulatory Frameworks (NERC CIP, State PUCs)
\item Data Privacy
\item Summary
\end{enumerate}

\item Measuring ROI and Business Value
\begin{enumerate}
\item The CPMAI Framework
\item Direct Value Metrics
\item Indirect Value Metrics
\item NPV and IRR Calculations
\item Case Studies
\item Summary
\end{enumerate}

\item Strategic Roadmap and Future Trends
\begin{enumerate}
\item Building an AI Strategy
\item Organizational Readiness
\item Technology Trends
\item Industry Evolution
\item Summary
\end{enumerate}

\item Epilogue: The Future of AI in Utilities
\begin{enumerate}
\item Looking Ahead
\item Continuous Learning
\item Summary
\end{enumerate}
\end{enumerate}

\subsection*{Part 6: Advanced Analytics}

\begin{enumerate}
\setcounter{enumi}{21}
\item Real-Time Analytics and Control Room Integration
\begin{enumerate}
\item Streaming Data Pipelines
\item Real-Time Anomaly Detection
\item Control Room Dashboards
\item Alert Generation
\item Summary
\end{enumerate}

\item Compliance Reporting and Reliability Metrics
\begin{enumerate}
\item SAIDI, SAIFI, and Other Reliability Metrics
\item Automated Reporting
\item Audit Trails
\item Summary
\end{enumerate}

\item Feature Engineering for Utility Data
\begin{enumerate}
\item Temporal Features
\item Geospatial Features
\item Grid Topology Features
\item Weather-Derived Features
\item Summary
\end{enumerate}

\item Reliability Analytics and Performance Metrics
\begin{enumerate}
\item Outage Cause Analysis
\item Predictive Reliability Models
\item Customer Impact Analysis
\item Summary
\end{enumerate}

\item Market Operations and Energy Trading
\begin{enumerate}
\item Price Forecasting
\item Bidding Optimization
\item Risk Analysis
\item Summary
\end{enumerate}

\item Causal Inference for Policy and Program Evaluation
\begin{enumerate}
\item Difference-in-Differences
\item Synthetic Control Method
\item Propensity Score Matching
\item Event Studies
\item Summary
\end{enumerate}

\item Multi-Task Learning for Utilities
\begin{enumerate}
\item Simultaneous Prediction of Multiple Outcomes
\item Hard Parameter Sharing
\item Single-Task Baselines
\item Applications to Emissions Prediction
\item Summary
\end{enumerate}
\end{enumerate}

\subsection*{Appendices}

\begin{enumerate}
\item[A] Installation and Setup
\begin{enumerate}
\item Python Environment Setup
\item Required Libraries
\item Data Access
\item Summary
\end{enumerate}

\item[B] Datasets and Data Sources
\begin{enumerate}
\item Public Datasets
\item Utility Data Formats
\item Data Quality Considerations
\item Summary
\end{enumerate}

\item[C] Troubleshooting Common Issues
\begin{enumerate}
\item Data Integration Challenges
\item Model Performance Issues
\item Deployment Problems
\item Summary
\end{enumerate}

\item[D] Regulatory Framework Overview
\begin{enumerate}
\item NERC CIP
\item State PUC Requirements
\item FERC Regulations
\item Summary
\end{enumerate}
\end{enumerate}

\end{document}

